\documentclass{article}
\usepackage{amsmath,amssymb}
\usepackage{graphicx}
\usepackage{subfigure}
\usepackage{url}
\usepackage{listings}

\title{Mini Project: Alternating Bit Protocol\footnote{This text is
adopted (and adapted) from the project description at the web page
of Luca Aceto.}}

\begin{document}

\maketitle

{\bf{Introduction to Concurrency and Verification, Winter
2007/2008\\

 \hspace*{3cm}\tt{www.cs.uni-salzburg.at/\~{ }anas}}}\\



The deadline for delivering this mini-project is Monday, 25
February, 2008. Deliver your solution via email. (You should receive
a confirmation email within two days. If not then resend your
solutions or contact me.) Pay special attention to the fact that the
solutions must be your own solutions and solutions of different
groups must be independent.

In this mini-project you are asked to model the alternating bit
protocol in the CCS language and verify it using CWB. The
alternating bit protocol is a simple yet effective protocol for
managing the retransmission of lost messages. Consider a sender S
and a receiver R, and assume that the communication medium from S to
R is initialized so that there are no messages in transit. The
alternating bit protocol works like this:

\begin{itemize}
\item Each message sent by S contains an additional protocol bit, 0
or 1.
\item When S sends a message, it sends it repeatedly (with its
corresponding bit) until receiving an acknowledgment (ACK) from R
that contains the same protocol bit as the message being sent.
\item When R receives a message, it sends an ACK to S and includes the
protocol bit of the message received. When a message is received for
the first time, the receiver delivers it for processing, while
subsequent messages with the same bit are simply acknowledged.
\item
When S receives an acknowledgment containing the same bit as the
message it is currently transmitting, it stops transmitting that
message, flips the protocol bit, and repeats the protocol for the
next message.
\end{itemize}

There is no direct communication between the sender and the
receiver; all messages must travel through the medium.

\paragraph{Your tasks are as follows:}

\begin{itemize}
\item Implement the alternating bit protocol in CWB. (Abstract away
from the content of the messages and focus only on the additional
control bit. To model the decision when the sender retransmits the
message, use either nondeterminism or ever better a special process
called Timer which will communicate with the sender on a channel
called timeout and thus signal when a message should be
retransmitted. You can also try to model the checksum verification -
see the link below - using nondeterminism.)
\item Suggest a
specification of the protocol and check whether it is equivalent to
your implementation (use a suitable equivalence notion available in
CWB). In particular, consider the following degrees of reliability
of the communication medium and answer this question for all of
these choices:
\begin{itemize}
          \item perfect channels (all received messages are delivered)
          \item lossy channels (received messages can get lost without any warning)
          \item lossy and duplicating (in addition the received message can be delivered several times).
\end{itemize}
\item Check for possible deadlocks (stuck configurations) and
livelocks (a possibility of an infinite sequence of tau actions) by
formulating the properties as recursive formulae and by verifying
whether the implementation satisfies these formulae.
\end{itemize}

\subsubsection*{How to deliver the mini-project?}



Create a short report. The report must be in pdf format and should
contain:
\begin{itemize}
          \item Full names and emails of all people that worked in your group (maximum 3 students per group).
          \item Your commented implementation and specification of the protocol (suggestion: use * for the comments in the protocol description and then copy/paste it directly into the report).
          \item A short conclusion about verification of the protocol using equivalence checking (including perfect, lossy, and lossy and duplicating channels). What equivalence did you choose and why?
          \item Formulae in CWB syntax for deadlock and livelock and whether your implementation does or does not satisfy these formulae.
          \item A short conclusion (e.g. Does the protocol satisfy the desired properties? What are the possible extensions of the protocol or do you have any suggestions for a more advanced modelling of some features of the protocol? What was your experience when working with CWB? ...).
      The report does not have to be long at all but it should contain (at least to some extent) all
      the points mentioned above. Send the report together with the CWB source file containing the
      implementation and specification of the protocol via email to me (ana.sokolova@cs.uni-salzburg.at).
\end{itemize}
Useful links:
\begin{itemize}
    \item A brief description of the protocol including checksum is at \\ {\tt{http://www.answers.com/topic/alternating-bit-protocol}}.
    \item A graphical simulation of the protocol is available at\\ {\tt{http://www.cs.stir.ac.uk/~kjt/software/comms/jasper/ABP.html}}.
    (Note that the control bits in the acknowledgment of the messages are switched.)
\end{itemize}

\end{document}
