\documentclass{article}
%\usepackage{entcsmacro}
\usepackage{graphicx}
\usepackage{amssymb,amsmath}

% The following is enclosed to allow easy detection of differences in
% ascii coding.
% Upper-case    A B C D E F G H I J K L M N O P Q R S T U V W X Y Z
% Lower-case    a b c d e f g h i j k l m n o p q r s t u v w x y z
% Digits        0 1 2 3 4 5 6 7 8 9
% Exclamation   !           Double quote "          Hash (number) #
% Dollar        $           Percent      %          Ampersand     &
% Acute accent  '           Left paren   (          Right paren   )
% Asterisk      *           Plus         +          Comma         ,
% Minus         -           Point        .          Solidus       /
% Colon         :           Semicolon    ;          Less than     <
% Equals        =3D           Greater than >          Question mark ?
% At            @           Left bracket [          Backslash     \
% Right bracket ]           Circumflex   ^          Underscore    _
% Grave accent  `           Left brace   {          Vertical bar  |
% Right brace   }           Tilde        ~

% A couple of exemplary definitions:

\newcommand{\Nat}{{\mathbb N}}
\newcommand{\Real}{{\mathbb R}}
\def\lastname{Hasuo, Jacobs, Sokolova}

\usepackage[all,2cell,dvips,ps]{xy}
% For faster compilation. Entries must be wrapped in curly braces!
\CompileMatrices \xyoption{v2} \xyoption{curve} \xyoption{2cell}
\SelectTips{cm}{}  % Tips (of arrows) are in accordance with Computer Modern
\UseAllTwocells \SilentMatrices
\def\labelstyle{\textstyle}
\def\twocellstyle{\textstyle}


\usepackage[curve]{xypic}


\newcommand{\Sets}{\mathbf{Sets}}
\newcommand{\setin}[3]{\{#1\in#2\;|\;#3\}}
\newcommand{\id}{\mathrm{id}}
\newcommand{\after}{\mathrel{\circ}}
\newcommand{\co}{\mathrel{\circ}}
\newcommand{\NNO}{{\mathbb{N}}}
\newcommand{\cat}[1]{{\mathbb{#1}}}
\newcommand{\congrightarrow}{\mathrel{\stackrel{
           \raisebox{.5ex}{$\scriptstyle\cong\,$}}{
           \raisebox{0ex}[0ex][0ex]{$\rightarrow$}}}}
\newcommand{\iso}{\congrightarrow}
\newcommand{\twocl}{\Rightarrow}
%\newcommand{\place}{\mbox{$-$}} % place holder
\newcommand{\place}{\underline{\phantom{n}}\,} % place holder
\newcommand{\pow}{\mathcal{P}}
\newcommand{\dist}{\mathcal{D}}
\newcommand{\lift}{\mathcal{L}}
\newcommand{\dcpo}{\mathbf{DCpo}}
\newcommand{\dcpob}{\mathbf{DCpo}_{\bot}}
\newdir{ >}{{}*!/-8pt/@{>}}  % for mono @{ >->}
\newcommand{\mono}{\rightarrowtail}
\newcommand{\epi}{\twoheadrightarrow}
\newcommand{\op}{\mathop{\mathrm{op}}\nolimits}
\newcommand{\weg}[1]{}
\newcommand{\defiff}{\;\stackrel{\mathrm{def}}
      {\Longleftrightarrow}\;}
\newcommand{\defeq}{\;\stackrel{\mathrm{def}}
      {=}\;}
\newcommand{\myQEDbox}{\Box}
\newcommand{\myQED}{\hspace*{\fill}$\myQEDbox$}
\newcommand{\st}{\mathsf{st}}
\newcommand{\dst}{\mathsf{dst}}
% for periods, base categories, etc.
% e.g. #1 = -3em, #2 = 1em, #3 = \Sets
\newcommand{\shifted}[3]{\save[]!<#1,#2>*{#3}\restore}
\newcommand{\idmap}[1]{\textrm{id}_{#1}}
\newcommand{\relliftop}[1]{\textrm{Rel}(#1)}
\newcommand{\rellift}[2]{\relliftop{#1}(#2)}
\newcommand{\Kleisli}[1]{\mathcal{K}{\kern-.2ex}\ell(#1)}
\newcommand{\trace}[1]{\mathsf{tr}_{#1}}
\newcommand{\toTerm}{\,\mathop{\text{\rm !}}\,}
\newcommand{\fromInit}{\,\mathop{\text{\rm \textexclamdown}}\,}
%\newcommand{\fromInit}{\rotatebox{180}{!}}

\DeclareMathOperator{\sync}{sync}

\usepackage{amsmath}
\usepackage{amssymb}
\usepackage{amstext}

\renewcommand{\arraystretch}{1.2}




\newif\ifignore % when set to true, additional text appears containing
                % further explanations or proofs (see \auxproof below)
%\ignoretrue
\ignorefalse
\newcommand{\auxproof}[1]{
\ifignore\mbox{}\newline \textbf{PROOF:} \dotfill\newline {\it
#1}\mbox{}\newline \textbf{ENDPROOF}\dotfill \fi}
\def\comment#1{\ifignore%
  \marginpar[\renewcommand{\baselinestretch}{0.9}\raggedleft\sloppy{}#1]%
    {\renewcommand{\baselinestretch}{0.9}\raggedright\sloppy{}#1}\fi}



\begin{document}

\title{Formale Systeme\\
 \small{{\bf{Example test 2, tasks to be discussed on January 30, 2020 }}} }
\date{}

\maketitle
\thispagestyle{empty}



\noindent{\bf{Task 1.}} (20 points) \quad Define the following notions:
\begin{itemize}
	\item[(a)] A relation $R$ is symmetric.
	\item[(b)] A relation $R$ is a total order.
	\item[(c)] A function $f$ is surjective.
	\item[(d)] $|A| = |B|$ for two sets $A$ and $B$.
\end{itemize}

\vspace*{5mm}

\noindent{\bf{Task 2.}} (20 points) \quad
Consider the relation $R$ on $\mathcal{P}(X)$ for a set $X$, defined by
$$(A,B) \in R \qquad \Leftrightarrow \qquad A \cup B \subseteq A .$$
Prove that $R$ is a transitive relation.\\

\noindent Hint: Maybe it helps if you reformulate the definition of $R$ using set properties.

\vspace*{5mm}


\noindent{\bf{Task 3.}} (20 points) \quad
Prove that the composition of two injective functions is an injective function, i.e., prove that if 
$f\colon A \to B$ and $g\colon B  \to C$ are both injective, then also $g \after f \colon A \to C$ is injective.


\vspace*{5mm}

\noindent{\bf{Task 4.}} (20 points) \quad
Let $B$ be an ordered set with partial order $\le$ on it.
Consider the set of all functions $$F_{A,B} = \{ f \mid f \colon A \to B \}.$$
We define a relation $\sqsubseteq$ on $F_{A,B}$ by (for $f, g \in F_{A,B}$):

$$ f \sqsubseteq g \qquad \Leftrightarrow \qquad \forall a \in A.\, f(a) \le g(a).$$  

Prove that $\sqsubseteq$ is a partial order too.

\vspace*{5mm}


\noindent{\bf{Task 5.}} (20 points) \quad
Construct a DFA for the language given by the regular expression 
$$(ab)^*(a \cup b)^*(baa^*)(abb^*).$$
Hint: It might be helpful if you first show that $(ab)^*(a \cup b)^* = (a \cup b)^*$.

\end{document}
